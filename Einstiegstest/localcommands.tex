\author{Prof.\ Dr.\ Roland Schäfer | Schwerpunkt \textit{Grammatik und Lexikon} | FSU Jena}
\title{Germanistische Sprachwissenschaft\\*%
  Einstiegstest Grammatik\ifdefined\SOLUTIONS\ \gruen{Musterlösung}\fi}

\date{Version Winter 2024 (\today)\\[0.5\baselineskip]
    \url{https://rolandschaefer.net/lehre-an-der-fsu-jena}}



\definecolor{rot}{rgb}{0.7,0.2,0.0}
\newcommand{\rot}[1]{\textcolor{rot}{#1}}
\definecolor{blau}{rgb}{0.1,0.2,0.7}
\newcommand{\blau}[1]{\textcolor{blau}{#1}}
\definecolor{gruen}{rgb}{0.0,0.7,0.2}
\newcommand{\gruen}[1]{\textcolor{gruen}{#1}}
\definecolor{grau}{rgb}{0.6,0.6,0.6}
\newcommand{\grau}[1]{\textcolor{grau}{#1}}
\definecolor{orongsch}{RGB}{255,165,0}
\newcommand{\orongsch}[1]{\textcolor{orongsch}{#1}}
\definecolor{tuerkis}{RGB}{63,136,143}
\definecolor{braun}{RGB}{108,71,65}
\newcommand{\tuerkis}[1]{\textcolor{tuerkis}{#1}}
\newcommand{\braun}[1]{\textcolor{braun}{#1}}

\newcommand*\Rot{\rotatebox{75}}

\newcommand{\zB}{z.\,B.\ }
\newcommand{\ZB}{Z.\,B.\ }
\newcommand{\Sub}[1]{\ensuremath{_{\text{#1}}}}
\newcommand{\Up}[1]{\ensuremath{^{\text{#1}}}}
\newcommand{\UpSub}[2]{\ensuremath{^{\text{#1}}_{\text{#2}}}}
\newcommand{\Doppelzeile}{\vspace{2\baselineskip}}
\newcommand{\Zeile}{\vspace{\baselineskip}}
\newcommand{\Halbzeile}{\vspace{0.5\baselineskip}}
\newcommand{\Viertelzeile}{\vspace{0.25\baselineskip}}

\newcommand{\whyte}[1]{\textcolor{white}{#1}}

\newcommand{\Spur}[1]{t\Sub{#1}}
\newcommand{\Ti}{\Spur{1}}
\newcommand{\Tii}{\Spur{2}}
\newcommand{\Tiii}{\Spur{3}}
\newcommand{\Tiv}{\Spur{4}}
\newcommand*{\mybox}[1]{\framebox{#1}}
\newcommand\ol[1]{{\setul{-0.9em}{}\ul{#1}}}

\newenvironment{nohyphens}{%
  \par
  \hyphenpenalty=10000
  \exhyphenpenalty=10000
  \sloppy
}{\par}

\newcommand{\Lf}{
  \setlength{\itemsep}{1pt}
  \setlength{\parskip}{0pt}
  \setlength{\parsep}{0pt}
}

\forestset{
  Ephr/.style={draw, ellipse, thick, inner sep=2pt},
  Eobl/.style={draw, rounded corners, inner sep=5pt},
  Eopt/.style={draw, rounded corners, densely dashed, inner sep=5pt},
  Erec/.style={draw, rounded corners, double, inner sep=5pt},
  Eoptrec/.style={draw, rounded corners, densely dashed, double, inner sep=5pt},
  Ehd/.style={rounded corners, fill=gray, inner sep=5pt,
    delay={content=\whyte{##1}}
  },
  Emult/.style={for children={no edge}, for tree={l sep=0pt}},
  phrasenschema/.style={for tree={l sep=2em, s sep=2em}},
  sake/.style={tier=preterminal},
  ake/.style={
    tier=preterminal
    },
}

\forestset{
  decide/.style={draw, chamfered rectangle, inner sep=2pt},
  finall/.style={rounded corners, fill=gray, text=white},
  intrme/.style={draw, rounded corners},
  yes/.style={edge label={node[near end, above, sloped, font=\scriptsize]{Ja}}},
  no/.style={edge label={node[near end, above, sloped, font=\scriptsize]{Nein}}}
}


\useforestlibrary{edges}

\forestset{
  narroof/.style={roof, inner xsep=-0.25em, rounded corners},
  forky/.style={forked edge, fork sep-=7.5pt},
  bluetree/.style={for tree={blau}, for children={edge=blau}},
  orongschtree/.style={for tree={orongsch}, for children={edge=orongsch}},
  rottree/.style={for tree={rot}, for children={edge=rot}},
  gruentree/.style={for tree={gruen}, for children={edge=gruen}},
  tuerkistree/.style={for tree={tuerkis}, for children={edge=tuerkis}},
  brauntree/.style={for tree={braun}, for children={edge=braun}}, 
  grautree/.style={for tree={grau}, for children={edge=grau}}, 
  gruennode/.style={gruen, edge=gruen},
  graunode/.style={grau, edge=grau},
  whitearc/.style={for children={edge=white}},
}

\defaultfontfeatures{Ligatures=TeX,Numbers=OldStyle, Scale=MatchLowercase}
\setmainfont{Linux Libertine O}
\setsansfont{Linux Biolinum O}

\setlength{\parindent}{0pt}

\newenvironment{spread}
{%
  \newdimen\origiwspc%
  \newdimen\origiwstr%
  \origiwspc=\fontdimen2\font%
  \origiwstr=\fontdimen3\font%
  \fontdimen2\font=1em%
  \doublespacing%
}{%
  \fontdimen2\font=\origiwspc%
  \fontdimen3\font=\origiwstr%
}


\newcommand{\Sol}[1]{%
  \ifdefined\SOLUTIONS%
    \gruen{#1}%
  \fi%
}

\newcommand{\SolX}[1]{%
  \ifdefined\SOLUTIONS%
    #1%
  \fi%
}

\newcommand{\Solalt}[3][gruen]{%
  \ifdefined\SOLUTIONS%
  \textcolor{#1}{#2}%
  \else%
    #3%
  \fi%
}

\newcommand{\SolaltX}[2]{%
  \ifdefined\SOLUTIONS%
    #1%
  \else%
    #2%
  \fi%
}

\newcommand{\Solul}[1]{\Solalt{\uline{#1}}{#1}}

\newcommand{\Solol}[1]{\Solalt{\ol{#1}}{#1}}

\newcommand{\Solframe}[1]{\Solalt{\mybox{#1}}{#1}}

\newcommand{\Solulmark}[1]{\Solalt{\uline{#1}}{#1}}

\newcommand{\Solbrack}[1]{\Solalt{[#1]}{#1}}

\newcommand{\Solmark}[2][gruen]{\Solalt{\textcolor{#1}{#2}}{#2}}

\newcommand{\Solcross}{\Solalt{\XBox}{\Square}}

\newcounter{aufgabe}
\newcommand{\aufgabeginn}{\setcounter{aufgabe}{1}}
\newcommand{\aufg}{(\arabic{aufgabe})\ \stepcounter{aufgabe}}


\newcommand{\praesenzaufgabe}[1][ | ]{\textbf{Präsenzaufgabe!}#1}
\newcommand{\morphologieaufgabe}[1][ | ]{\textbf{\tuerkis{Morphologie-Seminar}}#1}
\newcommand{\syntaxaufgabe}[1][ | ]{\textbf{\rot{Syntax-Vorlesung}}#1}
\newcommand{\graphematikaufgabe}[1][ | ]{\textbf{\blau{Graphematik-Vorlesung}}#1}
