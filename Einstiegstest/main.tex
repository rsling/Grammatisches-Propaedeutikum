\documentclass[12pt,a4paper,twoside]{article}

\usepackage[margin=2cm]{geometry}
\usepackage[ngerman]{babel}

\usepackage{setspace}
\usepackage{booktabs}
\usepackage{array,graphics}
\usepackage{color}
\usepackage{soul}
\usepackage[linguistics]{forest}
\usepackage{multirow}
\usepackage{longtable}
\usepackage{pifont}
\usepackage{wasysym}
\usepackage{langsci-gb4e}
\usepackage{soul}
\usepackage{enumitem}
%\usepackage{marginnote}
\usepackage{ulem}
\usepackage{hyperref}
\usepackage{tikz}
\usetikzlibrary{arrows,positioning} 
\usepackage{lineno}
\usepackage{fontspec}

\author{Prof.\ Dr.\ Roland Schäfer | Schwerpunkt \textit{Grammatik und Lexikon} | FSU Jena}
\title{Germanistische Sprachwissenschaft\\*%
  Einstiegstest Grammatik\ifdefined\SOLUTIONS\ \gruen{Musterlösung}\fi}

\date{Version Winter 2024 (\today)\\[0.5\baselineskip]
    \url{https://rolandschaefer.net/lehre-an-der-fsu-jena}}



\definecolor{rot}{rgb}{0.7,0.2,0.0}
\newcommand{\rot}[1]{\textcolor{rot}{#1}}
\definecolor{blau}{rgb}{0.1,0.2,0.7}
\newcommand{\blau}[1]{\textcolor{blau}{#1}}
\definecolor{gruen}{rgb}{0.0,0.7,0.2}
\newcommand{\gruen}[1]{\textcolor{gruen}{#1}}
\definecolor{grau}{rgb}{0.6,0.6,0.6}
\newcommand{\grau}[1]{\textcolor{grau}{#1}}
\definecolor{orongsch}{RGB}{255,165,0}
\newcommand{\orongsch}[1]{\textcolor{orongsch}{#1}}
\definecolor{tuerkis}{RGB}{63,136,143}
\definecolor{braun}{RGB}{108,71,65}
\newcommand{\tuerkis}[1]{\textcolor{tuerkis}{#1}}
\newcommand{\braun}[1]{\textcolor{braun}{#1}}

\newcommand*\Rot{\rotatebox{75}}

\newcommand{\zB}{z.\,B.\ }
\newcommand{\ZB}{Z.\,B.\ }
\newcommand{\Sub}[1]{\ensuremath{_{\text{#1}}}}
\newcommand{\Up}[1]{\ensuremath{^{\text{#1}}}}
\newcommand{\UpSub}[2]{\ensuremath{^{\text{#1}}_{\text{#2}}}}
\newcommand{\Doppelzeile}{\vspace{2\baselineskip}}
\newcommand{\Zeile}{\vspace{\baselineskip}}
\newcommand{\Halbzeile}{\vspace{0.5\baselineskip}}
\newcommand{\Viertelzeile}{\vspace{0.25\baselineskip}}

\newcommand{\whyte}[1]{\textcolor{white}{#1}}

\newcommand{\Spur}[1]{t\Sub{#1}}
\newcommand{\Ti}{\Spur{1}}
\newcommand{\Tii}{\Spur{2}}
\newcommand{\Tiii}{\Spur{3}}
\newcommand{\Tiv}{\Spur{4}}
\newcommand*{\mybox}[1]{\framebox{#1}}
\newcommand\ol[1]{{\setul{-0.9em}{}\ul{#1}}}

\newenvironment{nohyphens}{%
  \par
  \hyphenpenalty=10000
  \exhyphenpenalty=10000
  \sloppy
}{\par}

\newcommand{\Lf}{
  \setlength{\itemsep}{1pt}
  \setlength{\parskip}{0pt}
  \setlength{\parsep}{0pt}
}

\forestset{
  Ephr/.style={draw, ellipse, thick, inner sep=2pt},
  Eobl/.style={draw, rounded corners, inner sep=5pt},
  Eopt/.style={draw, rounded corners, densely dashed, inner sep=5pt},
  Erec/.style={draw, rounded corners, double, inner sep=5pt},
  Eoptrec/.style={draw, rounded corners, densely dashed, double, inner sep=5pt},
  Ehd/.style={rounded corners, fill=gray, inner sep=5pt,
    delay={content=\whyte{##1}}
  },
  Emult/.style={for children={no edge}, for tree={l sep=0pt}},
  phrasenschema/.style={for tree={l sep=2em, s sep=2em}},
  sake/.style={tier=preterminal},
  ake/.style={
    tier=preterminal
    },
}

\forestset{
  decide/.style={draw, chamfered rectangle, inner sep=2pt},
  finall/.style={rounded corners, fill=gray, text=white},
  intrme/.style={draw, rounded corners},
  yes/.style={edge label={node[near end, above, sloped, font=\scriptsize]{Ja}}},
  no/.style={edge label={node[near end, above, sloped, font=\scriptsize]{Nein}}}
}


\useforestlibrary{edges}

\forestset{
  narroof/.style={roof, inner xsep=-0.25em, rounded corners},
  forky/.style={forked edge, fork sep-=7.5pt},
  bluetree/.style={for tree={blau}, for children={edge=blau}},
  orongschtree/.style={for tree={orongsch}, for children={edge=orongsch}},
  rottree/.style={for tree={rot}, for children={edge=rot}},
  gruentree/.style={for tree={gruen}, for children={edge=gruen}},
  tuerkistree/.style={for tree={tuerkis}, for children={edge=tuerkis}},
  brauntree/.style={for tree={braun}, for children={edge=braun}}, 
  grautree/.style={for tree={grau}, for children={edge=grau}}, 
  gruennode/.style={gruen, edge=gruen},
  graunode/.style={grau, edge=grau},
  whitearc/.style={for children={edge=white}},
}

\defaultfontfeatures{Ligatures=TeX,Numbers=OldStyle, Scale=MatchLowercase}
\setmainfont{Linux Libertine O}
\setsansfont{Linux Biolinum O}

\setlength{\parindent}{0pt}

\newenvironment{spread}
{%
  \newdimen\origiwspc%
  \newdimen\origiwstr%
  \origiwspc=\fontdimen2\font%
  \origiwstr=\fontdimen3\font%
  \fontdimen2\font=1em%
  \doublespacing%
}{%
  \fontdimen2\font=\origiwspc%
  \fontdimen3\font=\origiwstr%
}


\newcommand{\Sol}[1]{%
  \ifdefined\SOLUTIONS%
    \gruen{#1}%
  \fi%
}

\newcommand{\SolX}[1]{%
  \ifdefined\SOLUTIONS%
    #1%
  \fi%
}

\newcommand{\Solalt}[3][gruen]{%
  \ifdefined\SOLUTIONS%
  \textcolor{#1}{#2}%
  \else%
    #3%
  \fi%
}

\newcommand{\SolaltX}[2]{%
  \ifdefined\SOLUTIONS%
    #1%
  \else%
    #2%
  \fi%
}

\newcommand{\Solul}[1]{\Solalt{\uline{#1}}{#1}}

\newcommand{\Solol}[1]{\Solalt{\ol{#1}}{#1}}

\newcommand{\Solframe}[1]{\Solalt{\mybox{#1}}{#1}}

\newcommand{\Solulmark}[1]{\Solalt{\uline{#1}}{#1}}

\newcommand{\Solbrack}[1]{\Solalt{[#1]}{#1}}

\newcommand{\Solmark}[2][gruen]{\Solalt{\textcolor{#1}{#2}}{#2}}

\newcommand{\Solcross}{\Solalt{\XBox}{\Square}}

\newcounter{aufgabe}
\newcommand{\aufgabeginn}{\setcounter{aufgabe}{1}}
\newcommand{\aufg}{(\arabic{aufgabe})\ \stepcounter{aufgabe}}


\newcommand{\praesenzaufgabe}[1][ | ]{\textbf{Präsenzaufgabe!}#1}
\newcommand{\morphologieaufgabe}[1][ | ]{\textbf{\tuerkis{Morphologie-Seminar}}#1}
\newcommand{\syntaxaufgabe}[1][ | ]{\textbf{\rot{Syntax-Vorlesung}}#1}
\newcommand{\graphematikaufgabe}[1][ | ]{\textbf{\blau{Graphematik-Vorlesung}}#1}


\begin{document}

\maketitle
\thispagestyle{empty}

\begin{tabular}[h]{lp{8cm}}
  \textbf{Name, Vorname} (nur falls gefordert) & \\\cline{2-2}
  &\\
  \textbf{Matrikelnummer} (nur falls gefordert) & \\\cline{2-2}
\end{tabular}

\vspace{2cm}

\begin{center}

  \Large{Dieser Test dient nur Ihrer eigenen Information.\\

    \Zeile

  Er zeigt Ihnen, welche Voraussetzungen für Ihr Studium\\
  im Bereich \textit{Grammatik} Sie bereits mitbringen und welche nicht.\\

  \Zeile

  Das hier getestete Wissen und die getesteten Fähigkeiten werden\\
  in der Schule oft auf unzulässig simple Weise dargestellt.\\
  Deswegen stellen sie \textbf{keine direkten Studieninhalte} dar,\\
  aber \textbf{wir rechnen trotzdem damit, dass Sie sie mitbringen}.\\
  
  \Zeile

  An der FSU Jena gilt: Zum Anfang des Morphologie-Seminars\\
  bearbeiten Sie alle Fragen, die mit \morphologieaufgabe[]\\
  gekennzeichnet sind, zum Anfang der Syntax-Vorlesung die\\
  mit \syntaxaufgabe[] gekennzeichneten. Für die\\
  \graphematikaufgabe[] gilt Paralleles.\\

  \Zeile

  \Sol{Achtung: Diese MuLö ist noch nicht geprüft!}
}
\end{center}

\newpage
\setcounter{page}{1}

\section{Wortarten im Deutschen}

\subsection{\morphologieaufgabe Klassifikation}

Die wichtigsten Wortarten des Deutschen sind die folgenden.
In runden Klammern steht jeweils eine übliche Abkürzung, in eckigen Klammern teilweise gebräuchliche \textbf{deutsche Namen der Wortklassen, die wir im Studium allerdings prinzipiell nicht verwenden}.
Im Studium werden diese Wortarten neu definiert, aber hier geht es erst einmal darum, zu sehen, ob Sie noch wissen, was in der Schule gelehrt wurde.

\begin{itemize}\Lf
  \item \textbf{Substantiv} (Subst) [Hauptwort, Dingwort, Gegenstandswort; auch oft (falsch): Nomen]
  \item \textbf{Adjektiv} (Adj) [Eigenschaftswort, Beiwort, Wie-Wort]
  \item \textbf{Artikel} (Art) [Geschlechtswort, Begleiter]
  \item \textbf{Pronomen} (Pro) [Fürwort]
  \item \textbf{Verb} (V) [Zeitwort, Tun-Wort]
  \item \textbf{Präposition} (Präp) [Beziehungswort, Verhältniswort]
  \item \textbf{Adverb} (Adv) [Umstandswort]
  \item \textbf{neben- und unterordnende Konjunktion} (NK, UK) [Bindewort]
  \item \textbf{Partikel} (Part)
\end{itemize}

Bestimmen Sie die Wortklassen im folgenden Kurztext, indem Sie die entsprechenden Abkürzungen unter die Wörter schreiben.
Gehen Sie dabei immer von der Wortklasse im gegebenen syntaktischen Kontext aus!
Für die ersten beiden Wörter wurde das beispielhaft schon erledigt.

\Zeile

\begin{center}
  \large
  \begin{tabular}[h]{|c|c|c|c|c|c|c|c|c|c|}
    \hline
      \textit{Ein} & \textit{Stuhl} & \textit{ist} & \textit{ein} & \textit{nützliches} & \textit{Möbelstück} & \textit{und} & \textit{dient} & \textit{dem} & \textit{Sitzen}. \\
      \hline
      Art  & Subst  & \Sol{V}  & \Sol{Art}  & \Sol{Adj}  & \Sol{Subst}  & \Sol{NK}  & \Sol{V}  & \Sol{Art} & \Sol{Subst} \\
    \hline
  \end{tabular}
\end{center}

\Zeile

\begin{center}
  \large
  \begin{tabular}[h]{|c|c|c|c|c|c|c|c|c|c|}
    \hline
      \textit{Oft} & \textit{steht} & \textit{vor} & \textit{ihm} & \textit{ein} & \textit{Tisch}, & \textit{dessen} & \textit{Beine} & \textit{länger} & \textit{sind}. \\
      \hline
      \Sol{Adv} & \Sol{V}  & \Sol{Präp}  & \Sol{Pro}  & \Sol{Art}  & \Sol{Subst}  & \Sol{Pro}  & \Sol{Subst}  & \Sol{Adj}  & \Sol{V} \\
    \hline
  \end{tabular}
\end{center}

\Zeile

\begin{center}
  \large
  \begin{tabular}[h]{|c|c|c|c|c|c|c|c|}
    \hline
      \textit{In} & \textit{Japan} & \textit{sehen} & \textit{traditionelle} & \textit{Tische} & \textit{ganz} & \textit{anders} & \textit{aus}, \\\hline
      \Sol{Präp} & \Sol{Subst}  & \Sol{V}  & \Sol{Adj}  & \Sol{Subst}  & \Sol{Adv}  & \Sol{Adj}  & \Sol{Part} \\\hline
      \textit{weil} & \textit{es} & \textit{dort} & \textit{ja} & \textit{auch} & \textit{keine} & \textit{Stühle} & \textit{gibt}. \\\hline
      \Sol{UK} & \Sol{Pro}  & \Sol{Adv}  & \Sol{Part}  & \Sol{Adv}  & \Sol{Art}  & \Sol{Subst}  & \Sol{V} \\
    \hline
  \end{tabular}
\end{center}

\newpage

\subsection{\morphologieaufgabe Substantiv}

Kreuzen Sie die korrekten Aussagen an.

\begin{itemize}[label=\Square]\Lf
  \item Im Plural sind alle Substantive grammatisch weiblich (z.\,B.\ \textit{der Tisch} \ding{222} \textit{die Tische}).
  \item[\Solcross] Alle Substantive sind entweder maskulin (grammatisch männlich), feminin (grammatisch weiblich) oder neutral (grammatisch sächlich).
  \item An allen Substantiven wird der Kasus (Fall) immer durch eine spezielle Endung angezeigt.
  \item[\Solcross] Fast alle Substantive haben für den Plural eine spezielle Form.
  \item[\Solcross] Man kann mehrere Substantive zu einem neuen Substantiv zusammensetzen.
  \item[\Solcross] Substantive sind nicht steigerbar.
  \item Substantive bezeichnen Dinge, die man anfassen kann.
  \item Maskuline Substantive können nur männliche Wesen bezeichnen.
  \item[\Solcross] Man kann mit zusätzlichen Endungen aus Verben und Adjektiven neue Substantive bilden.
\end{itemize}


\subsection{\morphologieaufgabe Adjektiv}

Kreuzen Sie die korrekten Aussagen an.

\begin{itemize}[label=\Square]\Lf
  \item Nach Adjektiven kann man immer mit \textit{Wie ist \ldots\ ?} fragen\\
    (z.\,B.: \textit{der rote Tisch} \ding{222} \textit{Wie ist der Tisch?} \ding{222} \textit{Rot.}).
  \item Adjektive haben ausnahmslos immer ein grammatisches Geschlecht (Genus).
    Das Geschlecht richtet sich nach einem Substantiv.
  \item Alle Adjektive bezeichnen Eigenschaften von Substantiven.
  \item[\Solcross] Adjektive haben besondere Formen, je nachdem, ob ein bestimmter oder unbestimmter Artikel vor ihnen steht.
  \item Adjektive sind inhaltlich ausschmückend und können daher immer weggelassen werden, ohne dass sich die Aussage des Satzes ändert.
  \item[\Solcross] Adjektive können auch wie Substantive verwendet werden, wenn kein Substantiv nach ihnen steht.
  \item[\Solcross] Prädikative Adjektive treten immer zusammen mit der Form eines Verbs wie \textit{sein}, \textit{bleiben}, \textit{werden} auf. \Sol{Auch ohne Kreuz korrekt, wenn man an Resultativprädikate oder sowas denkt.}
\end{itemize}


\subsection{\morphologieaufgabe Artikel}

Kreuzen Sie die korrekten Aussagen an.

\begin{itemize}[label=\Square]\Lf
  \item Artikel sind dazu da, das grammatische Geschlecht (Genus) des Substantivs anzuzeigen.
    Deswegen heißen sie in der Grundschuldidaktik (leicht veraltet) Geschlechtswort.
  \item[\Solcross] Artikel stehen immer vor einem Substantiv und stimmen mit diesem im Numerus (Singular\slash Plural) und dem Fall (Kasus) überein.
  \item[\Solcross] Alle Artikel haben jeweils spezifische Formen für die beiden Numeri (z.\,B.\ \textit{der Tisch} \ding{222} \textit{die Tische}).
\end{itemize}

% definit vs. indefinit

\newpage

\subsection{\morphologieaufgabe Pronomen}

Kreuzen Sie die korrekten Aussagen an.

\begin{itemize}[label=\Square]\Lf
  \item Pronomina ersetzen immer ein Substantiv.
  \item[\Solcross] Pronomina haben spezifische Formen für die Numeri (Singular\slash Plural).
  \item[\Solcross] Pronomina und Artikel sind die einzigen Wortklassen im Deutschen, an deren Mitgliedern man alle vier Kasus (Fälle) unterscheiden kann.
  \item Das Wort \textit{viel} wie in \textit{die vielen Erdbeeren} ist ein Indefinitpronomen.
  \item[\Solcross] Das Personalpronomen hat spezifische Formen für die drei grammatischen Personen im Singular.
  \item[\Solcross] Das Demonstrativpronomen hat spezifische Formen für die drei grammatischen Personen im Singular.
\end{itemize}


\subsection{\morphologieaufgabe Verb}

Kreuzen Sie die korrekten Aussagen an.

\begin{itemize}[label=\Square]\Lf
  \item[\Solcross] Starke Verben verändern in der einfachen Vergangenheitsform (Präteritum) ihren Vokal.
  \item Verben beschreiben immer Handlungen (\textit{essen}, \textit{kaufen}, \textit{vereinbaren} usw.).
  \item Verben müssen immer in ein Tempus (Zeitform) gesetzt werden (\textit{ich gehe}, \textit{ich ging} usw.).
  \item[\Solcross] Transitive Verben treten mit einem Subjekt und einem Akkusativobjekt auf.
  \item Nur transitive Verben kann man ins Passiv setzen\\
    (z.\,B.\ \textit{Wir kaufen den Saft. \ding{222}\ Der Saft wird gekauft.).}
  \item[\Solcross] Intransitive Verben haben kein Akkusativobjekt.
  \item[\Solcross] Das Verb \textit{sein} (\textit{ich bin} usw.) ist unregelmäßig.
  \item[\Solcross] Modalverben (\textit{müssen}, \textit{können} usw.) treten immer zusammen mit einem anderen Verb auf.
  \item Nach Verben kann man fragen mit \textit{Was macht\slash tut \ldots\ ?}
    Deswegen heißen sie in der Grundschuldidaktik Tun-Wörter.
  \item[\Solcross] Hilfsverben werden unter anderem benutzt, um Tempora (Zeitformen) auszudrücken.
  \item[\Solcross] Es gibt einen Infinitiv des Perfekts.
\end{itemize}


\subsection{\morphologieaufgabe Präposition}

Kreuzen Sie die korrekten Aussagen an.

\begin{itemize}[label=\Square]\Lf
  \item Präpositionen bestimmen ein Substantiv näher (z.\,B.\ \textit{unter dem Tisch}).
  \item[\Solcross] Präpositionen fordern immer einen bestimmten Kasus (Fall) beim Substantiv, das ihnen folgt.
  \item Präpositionen bilden immer adverbiale Bestimmungen und können weggelassen werden.
  \item[\Solcross] Manche Präpositionen können je nach Bedeutung entweder den Dativ oder den Akkusativ fordern.
\end{itemize}

\newpage

\subsection{\morphologieaufgabe Unterklassifikation von Verben}

Klassifizieren Sie die unterstrichenen Verben als starke Verben, schwache Verben, Modalverben oder Hilfsverben.

\begin{center}
  \resizebox{\textwidth}{!}{\begin{tabular}[h]{cp{0.5\textwidth}cccc}
    \toprule
    & \textbf{Verb im Satzkontext} & \multicolumn{4}{l}{\textbf{Bestimmung}} \\
    \midrule
    (1) & \textit{Marjella und ihre Freunde \ul{laufen} sehr schnell.}  & \Solcross~stark & \Square~schwach & \Square~Modalv. & \Square~Hilfsv. \\
    (2) & \textit{In den Urlaub \ul{wollten} 2020 viele fahren.}        & \Square~stark & \Square~schwach & \Solcross~Modalv. & \Square~Hilfsv. \\
    (3) & \textit{Wir \ul{kaufen} viel zu viel unnützes Zeug.}          & \Square~stark & \Solcross~schwach & \Square~Modalv. & \Square~Hilfsv. \\
    (4) & \textit{Du \ul{wirst} bald in den Urlaub fahren.}             & \Square~stark & \Square~schwach & \Square~Modalv. & \Solcross~Hilfsv. \\
    (5) & \textit{Es ist gut, dass sie wieder laufen \ul{kann}.}        & \Square~stark & \Square~schwach & \Solcross~Modalv. & \Square~Hilfsv. \\
    (6) & \textit{\ul{Durchschwimmen} kann man den Ärmelkanal auch.}    & \Solcross~stark & \Square~schwach & \Square~Modalv. & \Square~Hilfsv. \\
  \end{tabular}}
\end{center}

\Zeile

%%%%%%%%%%%%%%%%%%%%%%%%%%%%%%%%%%%%%%%%%%%%%%%%%%%%%%%%%%%%%%%%%%%%%%%
\section{Flexionskategorien deutscher Wörter}

\subsection{\morphologieaufgabe Flexion (Beugung)}

Bilden Sie die genannten Formen der unten in ihrer jeweiligen Nennform angegebenen Wörter.
Hinweis: Mit Präteritum bezeichnet man die einfache Vergangenheitsform.

\begin{center}
  \begin{tabular}[h]{cllp{0.3\textwidth}}
    \toprule
    & \textbf{Wort} & \textbf{zu bildende Form} & \textbf{Form} \\
    \midrule
    &&& \\
    (1) & \textit{fechten} & 3.~Person Singular Indikativ Präsens Aktiv    & \Sol{ficht} \\\cline{4-4}
    &&& \\
    (2) & \textit{Haus}    & Dativ Plural                                  & \Sol{Häusern} \\\cline{4-4}
    &&& \\
    (3) & \textit{laufen}  & 2.~Person Singular Indikativ Präteritum Aktiv & \Sol{liefst} \\\cline{4-4}
    &&& \\
    (4) & \textit{dies}    & Femininum Genitiv Singular                    & \Sol{dieser} \\\cline{4-4}
    &&& \\
    (5) & \textit{Oma}     & Genitiv Singular                              & \Sol{Oma (Oder doch \textit{Omas}?)} \\\cline{4-4}
    &&& \\
    (6) & \textit{streichen} & 3.~Person Plural Indikativ Futur 1 Passiv   & \Sol{werden gestrichen werden} \\\cline{4-4}
    \end{tabular}
\end{center}

\subsection{\morphologieaufgabe Kasus (Fall)}

Bestimmen Sie die Kasus -- also Nominativ, Akkusativ, Dativ, Genitiv -- der unterstrichenen Wörter.

\begin{center}
  \begin{tabular}[h]{cp{0.55\textwidth}cccc}
    \toprule
    & \textbf{Wort im Satzkontext} & \multicolumn{4}{l}{\textbf{Kasus}} \\
    \midrule
    (1) & \textit{\ul{Menschen} glauben wir oft zu leichtfertig.}      & \Square~Nom & \Square~Akk & \Solcross~Dat & \Square~Gen \\
    (2) & \textit{Günther lobt \ul{meinen} Fahrstil.}                  & \Square~Nom & \Solcross~Akk & \Square~Dat & \Square~Gen \\
    (3) & \textit{Selten wird das \ul{Auto} mehr als 200 km gefahren.} & \Solcross~Nom & \Square~Akk & \Square~Dat & \Square~Gen \\
    (4) & \textit{\ul{Es} wird deutlich zu viel Energie verbraucht.}   & \Solcross~Nom & \Square~Akk & \Square~Dat & \Square~Gen \\
    (5) & \textit{Das ist die Vorschrift, \ul{der} wir gehorchen.}     & \Square~Nom & \Square~Akk & \Solcross~Dat & \Square~Gen \\
    (6) & \textit{Das Auto der \ul{Kollegin} streikt mal wieder.}      & \Square~Nom & \Square~Akk & \Square~Dat & \Solcross~Gen \\
  \end{tabular}
\end{center}

\Zeile

\subsection{\morphologieaufgabe Genus (grammatisches Geschlecht)}

Bestimmen Sie das Genus -- also Maskulinum, Neutrum oder Femininum -- der unterstrichenen Wörter.

\begin{center}
  \begin{tabular}[h]{cp{0.55\textwidth}ccc}
    \toprule
    & \textbf{Wort im Satzkontext} & \multicolumn{3}{l}{\textbf{Kasus}} \\
    \midrule
    (1) & \textit{\ul{Der} Quark hält sich noch länger.}         & \Solcross~Mask & \Square~Neut & \Square~Fem \\
    (2) & \textit{\ul{Der} Kollegin gefällt das neue Büro.}      & \Square~Mask & \Square~Neut & \Solcross~Fem \\
    (3) & \textit{\ul{Der} Lämmer Fell ist weich.}               & \Square~Mask & \Solcross~Neut & \Square~Fem \\
    (4) & \textit{Dan sammelt kunstvolle \ul{Keramikkrüge}.}     & \Solcross~Mask & \Square~Neut & \Square~Fem \\
    (5) & \textit{Und reinigt die \ul{Tröge} gut!}               & \Solcross~Mask & \Square~Neut & \Square~Fem \\
    (6) & \textit{Wie diese \ul{Sykophanten} mal wieder nerven!} & \Solcross~Mask & \Square~Neut & \Square~Fem \\
  \end{tabular}
\end{center}


\subsection{\morphologieaufgabe Person}

Bestimmen Sie die Person -- also 1, 2 oder 3 -- der unterstrichenen Wörter (bzw. Wortgruppen im Fall von \textit{Herrn Gödel}).


\begin{center}
  \begin{tabular}[h]{cp{0.55\textwidth}ccc}
    \toprule
    & \textbf{Wort im Satzkontext} & \multicolumn{3}{l}{\textbf{Person}} \\
    \midrule
    (1) & \textit{Ich wünschte, du \ul{höbst} den Schwamm auf.}               & \Square~1 & \Solcross~2 & \Square~3 \\
    (2) & \textit{Mir gefällt \ul{euer} Haus sehr.}                           & \Square~1 & \Square~2 & \Solcross~3 \\
    (3) & \textit{Ich sehe \ul{es}.}                                          & \Square~1 & \Square~2 & \Solcross~3 \\
    (4) & \textit{Meine ehemalige Kollegin heißt \ul{Marjella}}               & \Square~1 & \Square~2 & \Solcross~3 \\
    (5) & \textit{\ul{Ich} bin Elektrotechniker.}                             & \Solcross~1 & \Square~2 & \Square~3 \\
    (6) & \textit{Sein Genie hat \ul{Herrn Gödel} den Verstand gekostet.}     & \Square~1 & \Square~2 & \Solcross~3 \\
    (7) & \textit{Dass Strom billiger würde, \ul{möchte} mir schon gefallen.} & \Square~1 & \Square~2 & \Solcross~3 \\
  \end{tabular}
\end{center}

\enlargethispage{-2\baselineskip}

\subsection{\morphologieaufgabe Finitheit}

\ul{Unterstreichen} Sie im folgenden Text alle finiten Verbformen und \mybox{rahmen} Sie alle infiniten Verbformen ein.
Als infinite Verbformen sollen hier auch Partizipien in adjektivischer Funktion usw.\ gelten.

\begin{quote}
  \begin{spread}
  \it Die Sokal-Affäre (auch Sokal-Debatte oder Sokal-Kontroverse) \Solul{war} eine Auseinandersetzung über die intellektuellen Standards in den Sozial- und Geisteswissenschaften, die durch die Veröffentlichung eines Hoax-Artikels des Physikers Alan Sokal in der sozialwissenschaftlichen Fachzeitschrift Social Text \Solframe{ausgelöst} \Solul{wurde}. Sokals Artikel \Solul{erschien} 1996 in einer den Science Wars (Wissenschaftskriegen) \Solframe{gewidmeten} Ausgabe, die die US-spezifische Auseinandersetzung zwischen wissenschaftlichem Realismus und Postmoderne \Solframe{thematisieren} \Solul{sollte}.\\
  Sokals Beitrag \Solul{war} in postmodernem Jargon \Solframe{formuliert} und \Solul{gab} vor, die Quantengravitation als linguistisches und soziales Konstrukt \Solframe{zu deuten}, wobei die Quantenphysik die postmodernistische Kritik \Solul{stütze}. Sokal \Solul{hatte} dabei absichtlich zahlreiche logische und inhaltliche Fehler \Solframe{eingestreut}, die den Redakteuren der Zeitschrift – sie \Solul{hatten} für die Schlussredaktion keine Physikexperten \Solframe{hinzugezogen} – jedoch nicht \Solul{auffielen}. Es \Solul{folgte} eine wissenschaftstheoretische und öffentliche Debatte über \Solframe{mangelnde} intellektuelle Strenge bei der Bewertung pseudowissenschaftlicher Artikel in den Sozial- und Geisteswissenschaften und einen möglicherweise schädlichen Einfluss postmoderner Philosophie auf diese Wissenschaften. Weiterhin \Solul{wurde} diesen Disziplinen \Solframe{vorgeworfen}, naturwissenschaftliche Konzepte in sinnloser oder missbräuchlicher Weise für ihre Lehren \Solframe{zu verwenden}.

  \footnotesize{[Quelle: https://de.wikipedia.org/wiki/Sokal-Affäre, modifiziert]}
  \end{spread}
\end{quote}


\subsection{\morphologieaufgabe Tempus (Zeitform)}

Bestimmen Sie das Tempus der folgenden Sätze.
Die Tempuskategorien im weiteren Sinn sind im Deutschen: \textbf{Präsens} (Gegenwart), \textbf{Präteritum} (einfache Vergangenheit), \textbf{Perfekt} (vollendete Vergangenheit), \textbf{Plusquamperfekt} (Vorvergangenheit), \textbf{Futur 1} (einfache Zukunft) und \textbf{Futur 2} (Futurum Exactum oder Vergangenheit in der Zukunft).
Verwenden Sie ggf.\ die Abkürzungen \textbf{Präs}, \textbf{Prät}, \textbf{Perf}, \textbf{Plusq}, \textbf{F1}, \textbf{F2}.

\begin{center}
  \begin{tabular}[h]{clp{0.3\textwidth}}
    \toprule
    & \textbf{Wort im Satzkontext} & \textbf{Tempus} \\
    \midrule
    && \\
    (1) & \textit{Der Schwan wird ungern fotografiert.}                     & \Sol{Präs} \\ \cline{3-3}
    && \\
    (2) & \textit{Mir war übel geworden.}                                   & \Sol{Plusq} \\ \cline{3-3}
    && \\
    (3) & \textit{Im  Jahr 2050 wird Helmut Schmidt abgewählt worden sein.} & \Sol{F2} \\ \cline{3-3}
    && \\
    (4) & \textit{Morgen gehe ich endlich zur Post!}                        & \Sol{Präs} \\ \cline{3-3}
    && \\
    (5) & \textit{Der Schwan wird ungern fotografiert werden.}              & \Sol{Fut} \\ \cline{3-3}
    && \\
    (6) & \textit{Das Boot hat auslaufen können.}                           & \Sol{Perf} \\ \cline{3-3}
    && \\
    (7) & \textit{1993 hat der Kommerz den Techno zerstört.}                & \Sol{Perf} \\ \cline{3-3}
  \end{tabular}
\end{center}

\newpage

\subsection{\morphologieaufgabe Modus (Aussageweise)}

Setzen Sie den folgenden kurzen Absatz zunächst in den Konjunktiv 1, dann in den Konjunktiv 2.
Folgen Sie den Normregeln für die Ersetzung von Formen, die ansonsten nicht eindeutig wären.

\begin{quote}
  Die Grammatik folgt Regeln, und sie folgte schon immer Regeln.
  Nur das kann der Grund sein, dass wir einander verstehen, wenn wir Sprache benutzen.
  Die Mathematik ist axiomatisch eingeführt worden.
  Sie gehorcht damit ausnahmslosen Regeln, während die Regeln der Grammatik Ausnahmen zulassen.
\end{quote}

\textbf{Im Konjunktiv 1:}

  \begin{tabular}[h]{p{1\textwidth}}
  \\
  \Sol{Die Grammatik folge regeln, und sie sei schon immer Regeln gefolgt. Nur das}\\\hline
  \\
  \Sol{könne der Grund sein, dass wir einander verstünden, wenn wir Sprache}\\\hline
  \\
  \Sol{benutzten. Die Mathematik sei axiomatisch eingeführt worden. Sie gehorche}\\\hline
  \\
  \Sol{damit ausnahmslosen Regeln, während die Regeln der Grammatik Ausnahmen}\\\hline
  \\
  \Sol{zuließen.}\\\hline
  \\
  \\\hline
  \end{tabular}

  \Doppelzeile

\textbf{Im Konjunktiv 2:}

  \begin{tabular}[h]{p{1\textwidth}}
  \\
  \Sol{Die Grammatik würde Regeln folgen, und sie wäre schon immer Regeln gefolgt.}\\\hline
  \\
  \Sol{Nur das könnte der Grund sein, dass wir einander verstünden, wenn wir}\\\hline
  \\
  \Sol{Sprache benutzen würden. Die Mathematik wäre axiomatisch eingeführt worden.}\\\hline
  \\
  \Sol{Sie würde damit ausnahmslosen Regeln gehorchen, während die Regeln der}\\\hline
  \\
  \Sol{Grammatik Ausnahmen zuließen.}\\\hline
  \\
  \\\hline
  \end{tabular}

\newpage

\subsection{\morphologieaufgabe Imperativ (Aufforderungsform)}

Kreuzen Sie die Sätze an, in denen eine echte Imperativform vorkommt.

\begin{itemize}[label=\Square]\Lf
  \item[\Solcross] \textit{Komm du mir nur nach hause!}
  \item \textit{Ich möchte, dass du sofort dein Zimmer aufräumst.}
  \item \textit{Den Eischnee langsam mit einer Gabel unterheben.}
  \item \textit{Wirst du wohl die anderen Kinder in Ruhe lassen!}
  \item[\Solcross] \textit{Nimm dir bitte Schokolade, soviel du möchtest.}
  \item \textit{Stehengeblieben!}
  \item \textit{Hier wird nicht geraucht!}
  \item \textit{Ich befehle dir, sofort mit dem Hund rauszugehen.}
  \item \textit{Liegenbleiben!}
  \item[\Solcross] \textit{Glaubt bloß nicht, dass die Krankenkasse das bezahlt.}
  \item \textit{Gibst du das wohl sofort deiner Schwester zurück!}
\end{itemize}



\subsection{\morphologieaufgabe Genus verbi (Aktiv\slash Passiv)}

Setzen Sie die folgenden Sätze ins Aktiv, wenn es Passivsätze sind, und ins Passiv, wenn es Aktivsätze sind.

\begin{center}
  \begin{longtable}[h]{cp{0.9\textwidth}}
    & \\
    (1) & \textit{Ein Kollege gibt mir das Buch.} \\
    & \\
    & \Sol{Das Buch wird mir (von einem Kollegen) gegeben.} \\\cline{2-2}
    & \\
    & \\\cline{2-2}
    & \\
    (2) & \textit{Der Kuchen wurde von unserem Hund gegessen.} \\
    & \\
    & \Sol{Unser Hund aß den Kuchen.} \\\cline{2-2}
    & \\
    &\\\cline{2-2}
    & \\
    & \\
    (3) & \textit{Der Kuchen ist von unserem Hund gegessen worden.} \\
    & \\
    & \Sol{Unser Hund hat den Kuchen gegessen.} \\\cline{2-2}
    & \\
    &\\\cline{2-2}
    & \\
    (4) & \textit{Man kauft hier gerne Limo.} \\
    & \\
    & \Sol{Limo wird hier gerne gekauft.} \\\cline{2-2}
    & \\
    &\\\cline{2-2}
    & \\
    (5) & \textit{Hier wird nicht geraucht!} \\*
    & \\*
    &\Sol{Man raucht hier nicht!}\\*\cline{2-2}
    & \\*
    &\\*\cline{2-2}
  \end{longtable}
\end{center}

\vspace{-2\baselineskip}

\section{Satzbau (Syntax)}


\subsection{\syntaxaufgabe Satzglieder}

Zeichnen Sie einen \mybox{Kasten} um jedes Satzglied in folgenden Sätzen.

\begin{exe}
  \setcounter{xnumi}{0}
  \ex \textit{\Solframe{Menschen} glauben \Solframe{wir} \Solframe{oft} \Solframe{zu leichtfertig}.}\\

  \ex \textit{\Solframe{Günther} lobt \Solframe{meinen Fahrstil}.}\\

  \ex \textit{\Solframe{Selten} wird \Solframe{das Auto} \Solframe{mehr als 200 km} gefahren.}\\

  \ex \textit{\Solframe{Es} wird \Solframe{deutlich zu viel Energie} verbraucht.}\\

  \ex \textit{\Solframe{Das} ist \Solframe{die Vorschrift, der wir gehorchen}.}\\

  \ex \textit{\Solframe{Das Auto der Kollegin} streikt \Solframe{mal wieder}.}
\end{exe}


\subsection{\syntaxaufgabe Subjekt}

Unterstreichen Sie das Subjekt in den folgenden Sätzen.

\begin{exe}
  \setcounter{xnumi}{0}
  \ex Dass die Welt vergänglich ist, weiß \Solul{ich}.\\

    \ex Gestern hatte \Solul{der Kollege} das Buch noch gesehen.\\

    \ex \Solul{Dass die Welt vergänglich ist}, ist mir bekannt.\\

    \ex Es gehen mir hier \Solul{zu viele Leute} über die Straße.\\

    \ex \Solul{Den Mülleimer zu leeren}, nervt Matthias.\\

    \ex Uns graut vor den neuen Quartalszahlen. \Sol{Kein Subjekt!}\\

  \ex Das Auto fährt mir \Solul{die Oma} zu oft zu schnell.\\

\end{exe}

\newpage

\subsection{\syntaxaufgabe Objekte und adverbiale Bestimmungen}

\textbf{\ul{Unterstreichen}} Sie im folgenden Text die direkten Objekte in den folgenden Sätzen und \textbf{\ol{überstreichen}} Sie in denselben Sätzen alle indirekten Objekte. Die Präpositionalobjekte \textbf{\mybox{rahmen}} Sie ein.
Die adverbialen Bestimmungen \textbf{[klammern]} Sie bitte ein.

\vspace{-1\baselineskip}

\begin{quote}
  \begin{spread}
    Die Schlacht von Worringen war 1288 das kriegerische Finale \Solbrack{im zuvor bereits sechs Jahre währenden Limburger Erbfolgestreit}.
  Hauptkontrahenten des Konflikts waren Siegfried von Westerburg, Erzbischof von Köln, und Herzog Johann I. von Brabant.
  Der Ausgang der Schlacht veränderte \Solul{das Machtgefüge im gesamten Nordwesten Mitteleuropas}.\\
  Der Ausgang der Schlacht hatte für jede der involvierten Parteien \Solul{er\-heb\-liche Konsequenzen}.
  Erzbischof Siegfried von Westerburg befand sich als Gefangener \Solframe{in der Gewalt des Grafen von Berg} \Solbrack{im "`Novum Cas\-trum"'}.
  \Solbrack{Erst durch den Sühnevertrag vom 19. Mai 1289} erlangte er \Solul{die Freiheit} wieder.
  Inzwischen hatte der Dompropst von {\tiny Köln}, Konrad von Berg, ein Bruder von Adolf von Berg, \Solul{die Regierungsgewalt des Erzstifts} übernommen.
  Die Gewinner der Schlacht hatten \Solul{Tatsachen} geschaffen, \Solul{die} Siegfried \Solbrack{neben der Lösegeldzahlung von 12.000 Mark} \Solbrack{wohl oder übel} \Solbrack{durch den Sühnevertrag} billigen musste.
  \Solbrack{Außerdem} musste er \Solframe{auf sein Befestigungsrecht im Bergischen Land} verzichten.
  Eberhard von der Mark erhielt \Solul{Befestigungshoheit} und Adolf von Berg \Solul{sein Münzrecht, auf das er 1279 zugunsten des Erzbischofs hatte verzichten müssen}, zurück.

  \footnotesize{[Quelle: https://de.wikipedia.org/wiki/Schlacht\_von\_Worringen, modifiziert]}
  \end{spread}
\end{quote}


\newpage

\subsection{\syntaxaufgabe Nebensätze}

Bestimmen Sie die Nebensätze in den folgenden Sätzen als Subjektsatz, Objektsatz, Adverbialsatz oder Relativsatz.

\begin{center}
  \begin{tabular}[h]{cp{0.5\textwidth}cccc}
    \toprule
    & \textbf{Satz mit Nebensatz} & \multicolumn{4}{l}{\textbf{Nebensatzart}} \\
    \midrule
    (1) & \textit{Damit es nicht zu spät wird, gehen wir jetzt.}    & \Square~Subj & \Square~Obj & \Solcross~Adv & \Square~Rel \\
    (1) & \textit{Wer das glaubt, hat keine Ahnung von Physik.}     & \Square~Subj & \Square~Obj & \Square~Adv & \Solcross~Rel \\
    (1) & \textit{Ob die Sonne scheinen wird, ist die große Frage.} & \Solcross~Subj & \Square~Obj & \Square~Adv & \Square~Rel \\
    (1) & \textit{Marjella freut, dass die Sonne scheint.}          & \Solcross~Subj & \Square~Obj & \Square~Adv & \Square~Rel \\
    (1) & \textit{Wir fragen uns, ob das Wetter heute gut wird.}    & \Square~Subj & \Solcross~Obj & \Square~Adv & \Square~Rel \\
    (1) & \textit{Das ist der Kollege, dessentwegen ich hier bin.}  & \Square~Subj & \Square~Obj & \Square~Adv & \Solcross~Rel \\
  \end{tabular}
\end{center}


\Zeile

\subsection{\syntaxaufgabe Infinitivgruppen}

Unterstreichen Sie die kompletten Infinitivgruppen (auch: erweiterte Infinitive) in folgenden Sätzen.
Bestimmen Sie außerdem ihre Funktion im Satz: Subjekt, Objekt oder Adverbial.

\begin{center}
  \resizebox{\textwidth}{!}{\begin{tabular}[h]{cp{0.65\textwidth}ccc}
    \toprule
    & \textbf{Infinitivgr.\ im Satzkontext} & \multicolumn{3}{l}{\textbf{Bestimmung}} \\
    \midrule
    (1) & \textit{\Solul{Den Stuhl zu reparieren}, mag Matthias nicht.}                & \Square~Subj & \Solcross~Obj & \Square~Adv \\
    &&&& \\
    (2) & \textit{\Solul{Den Stuhl zu reparieren}, nervt.}                             & \Solcross~Subj & \Square~Obj & \Square~Adv \\
    &&&& \\
    (3) & \textit{\Solul{Um den Stuhl zu reparieren}, geht Matthias in die Werkstatt.} & \Square~Subj & \Square~Obj & \Solcross~Adv \\
    &&&& \\
    (4) & \textit{Sie wagt, \Solul{die Küche zu betreten}.}                            & \Square~Subj & \Solcross~Obj & \Square~Adv \\
    &&&& \\
    (5) & \textit{Er stellt die Karre ab, \Solul{ohne den Lackschaden zu erwähnen}.}   & \Square~Subj & \Square~Obj & \Solcross~Adv \\
    &&&& \\
    (6) & \textit{Es wurde versucht, \Solul{die Demonstration zu verhindern}.}         & \Solcross~Subj & \Square~Obj & \Square~Adv \\
    &&&& \\
  \end{tabular}}
\end{center}


\newpage

% %%%%%%%%%%%%%%%%%%%%%%%%%%%%%%%%%%%%%%%%%%%%%%%%%%%%%%%%%%%%%%%%%%%%%%%
\section{Buchstaben und Laute (Graphematik)}
 
\subsection{\graphematikaufgabe Laute und Buchstaben}

Welche der unterstrichenen Buchstaben oder Buchstabengruppen in den folgenden Wortpaaren werden in beiden Wörtern gleich ausgesprochen?

\begin{center}
  \begin{tabular}[h]{rllcc}
    \toprule
    & \textbf{Wort 1} & \textbf{Wort 2} & \multicolumn{2}{l}{\textbf{Aussprache}} \\
    \midrule
    (1) & ba\ul{t}      & Ba\ul{d}         & \Solcross~gleich & \Square~nicht gleich \\
    (2) & wei\ul{ch}en  & wa\ul{ch}en      & \Square~gleich & \Solcross~nicht gleich \\
    (3) & R\ul{o}be     & R\ul{o}bbe       & \Square~gleich & \Solcross~nicht gleich \\
    (4) & \ul{k}lein    & ha\ul{ck}en      & \Solcross~gleich & \Square~nicht gleich \\ 
    (5) & \ul{L}and     & Ba\ul{ll}        & \Solcross~gleich & \Square~nicht gleich \\
    (6) & sp\ul{ä}ter   & \ul{Eh}re        & \Square~gleich & \Solcross~nicht gleich \\
    (7) & kl\ul{ar}     & F\ul{ah}ne       & \Square~gleich & \Solcross~nicht gleich \\
    (8) & \ul{r}ar      & ra\ul{r}         & \Square~gleich & \Solcross~nicht gleich \\
    (9) & R\ul{eh}      & Schn\ul{ee}      & \Solcross~gleich & \Square~nicht gleich \\
    (10) & frü\ul{h}er  & \ul{h}art        & \Square~gleich & \Solcross~nicht gleich \\
  \end{tabular}
\end{center}


\subsection{\graphematikaufgabe Silben}\label{sec:silben}

Trennen Sie die folgenden Wörter in Silben.
Nutzen Sie dazu wie in Beispiel (0) demonstriert Punkte als Trenner.
Für das erste Wort gibt es eine Lösung als Beispiel.

\begin{center}
  \begin{longtable}[h]{clp{0.5\textwidth}}
    && \\
    (0) & \textit{Tinte} & Tin.te \\ \cline{3-3}
    && \\
    (1) & \textit{verwundert} & \Sol{ver.wún.dert} \\ \cline{3-3}
    && \\
    (2) & \textit{Desorientierung} & \Sol{Des.o.ri.en.tíe.rung} \\ \cline{3-3}
    && \\
    (3) & \textit{Wege} & \Sol{W\'e.ge} \\ \cline{3-3}
    && \\
    (4) & \textit{Automat} & \Sol{Au.to.mát} \\ \cline{3-3}
    && \\
    (5) & \textit{Anklang} & \Sol{Án.klang} \\ \cline{3-3}
    && \\
    (6) & \textit{Politik} & \Sol{Po.li.tík} \\ \cline{3-3}
    && \\
    (7) & \textit{Iglo} & \Sol{Í.glo} \\ \cline{3-3}
    && \\
    (8) & \textit{Anschrift} & \Sol{Án.schrift} \\ \cline{3-3}
    && \\
    (9) & \textit{Küchen} & \Sol{K\'{ü}\ul{ch}en (Silbengelenk)} \\ \cline{3-3}
    && \\
    (10) & \textit{munter} & \Sol{mún.ter} \\ \cline{3-3}
    && \\
    (11) & \textit{strolchtest} & \Sol{strólch.test} \\ \cline{3-3}
    && \\
    (12) & \textit{klapprigstes} & \Sol{klápp.rig.stes} \\ \cline{3-3}
    && \\
    (13) & \textit{Marmelade} & \Sol{Mar.me.lá.de} \\ \cline{3-3}
    && \\
    (14) & \textit{Mangel} & \Sol{Má\ul{ng}el (Silbengelenk)} \\ \cline{3-3}
    && \\
    (15) & \textit{Metropolis} & \Sol{Me.tró.po.lis} \\ \cline{3-3}
  \end{longtable}
\end{center}

\subsection{\graphematikaufgabe Betonung}

Setzen Sie in Aufgabe~\ref{sec:silben} einen Akutakzent (also das Zeichen \'\ \,) über den Vokal der betonten Silbe in den von Ihnen in Silben zerlegten Wörtern in Aufgabe~\ref{sec:silben}.
Also \textbf{T\'in.te} usw.

\end{document}
