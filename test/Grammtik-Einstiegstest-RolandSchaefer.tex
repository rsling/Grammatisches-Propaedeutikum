\documentclass[12pt,a4paper,twoside]{article}

\usepackage[margin=2cm]{geometry}

\usepackage[ngerman]{babel}

\usepackage{setspace}
\usepackage{booktabs}
\usepackage{array,graphics}
\usepackage{color}
\usepackage{soul}
\usepackage[linecolor=gray,backgroundcolor=yellow!50,textsize=tiny]{todonotes}
\usepackage[linguistics]{forest}
\usepackage{multirow}
\usepackage{pifont}
\usepackage{wasysym}
\usepackage{langsci-gb4e}
\usepackage{soul}
\usepackage{enumitem}
\usepackage{marginnote}

\usepackage[maxbibnames=99,
  maxcitenames=2,
  uniquelist=false,
  backend=biber,
  doi=false,
  url=false,
  isbn=false,
  bibstyle=biblatex-sp-unified,
  citestyle=sp-authoryear-comp]{biblatex}

\definecolor{rot}{rgb}{0.7,0.2,0.0}
\newcommand{\rot}[1]{\textcolor{rot}{#1}}
\definecolor{blau}{rgb}{0.1,0.2,0.7}
\newcommand{\blau}[1]{\textcolor{blau}{#1}}
\definecolor{gruen}{rgb}{0.0,0.7,0.2}
\newcommand{\gruen}[1]{\textcolor{gruen}{#1}}
\definecolor{grau}{rgb}{0.6,0.6,0.6}
\newcommand{\grau}[1]{\textcolor{grau}{#1}}
\definecolor{orongsch}{RGB}{255,165,0}
\newcommand{\orongsch}[1]{\textcolor{orongsch}{#1}}
\definecolor{tuerkis}{RGB}{63,136,143}
\definecolor{braun}{RGB}{108,71,65}
\newcommand{\tuerkis}[1]{\textcolor{tuerkis}{#1}}
\newcommand{\braun}[1]{\textcolor{braun}{#1}}

\newcommand*\Rot{\rotatebox{75}}

\newcommand{\Sub}[1]{\ensuremath{_{\text{#1}}}}
\newcommand{\Up}[1]{\ensuremath{^{\text{#1}}}}
\newcommand{\UpSub}[2]{\ensuremath{^{\text{#1}}_{\text{#2}}}}
\newcommand{\Doppelzeile}{\vspace{2\baselineskip}}
\newcommand{\Zeile}{\vspace{\baselineskip}}
\newcommand{\Halbzeile}{\vspace{0.5\baselineskip}}
\newcommand{\Viertelzeile}{\vspace{0.25\baselineskip}}

\newcommand{\whyte}[1]{\textcolor{white}{#1}}

\newcommand{\Spur}[1]{t\Sub{#1}}
\newcommand{\Ti}{\Spur{1}}
\newcommand{\Tii}{\Spur{2}}
\newcommand{\Tiii}{\Spur{3}}
\newcommand{\Tiv}{\Spur{4}}
\newcommand*{\mybox}[1]{\framebox{#1}}
\newcommand\ol[1]{{\setul{-0.9em}{}\ul{#1}}}

\newcommand{\Lf}{
  \setlength{\itemsep}{1pt}
  \setlength{\parskip}{0pt}
  \setlength{\parsep}{0pt}
}

\forestset{
  Ephr/.style={draw, ellipse, thick, inner sep=2pt},
  Eobl/.style={draw, rounded corners, inner sep=5pt},
  Eopt/.style={draw, rounded corners, densely dashed, inner sep=5pt},
  Erec/.style={draw, rounded corners, double, inner sep=5pt},
  Eoptrec/.style={draw, rounded corners, densely dashed, double, inner sep=5pt},
  Ehd/.style={rounded corners, fill=gray, inner sep=5pt,
    delay={content=\whyte{##1}}
  },
  Emult/.style={for children={no edge}, for tree={l sep=0pt}},
  phrasenschema/.style={for tree={l sep=2em, s sep=2em}},
  sake/.style={tier=preterminal},
  ake/.style={
    tier=preterminal
    },
}

\forestset{
  decide/.style={draw, chamfered rectangle, inner sep=2pt},
  finall/.style={rounded corners, fill=gray, text=white},
  intrme/.style={draw, rounded corners},
  yes/.style={edge label={node[near end, above, sloped, font=\scriptsize]{Ja}}},
  no/.style={edge label={node[near end, above, sloped, font=\scriptsize]{Nein}}}
}

\usepackage{tikz}
\usetikzlibrary{arrows,positioning} 


\author{Prof.\ Dr.\ Roland Schäfer | Schwerpunkt \textit{Grammatik und Lexikon}}
\title{FSU Jena | Germanistische Sprachwissenschaft\\Einstiegstest Grammatik}
\date{Version Sommer 2023 (\today)}


\usepackage{fontspec}
\defaultfontfeatures{Ligatures=TeX,Numbers=OldStyle, Scale=MatchLowercase}
\setmainfont{Linux Libertine O}
\setsansfont{Linux Biolinum O}

\setlength{\parindent}{0pt}

\usepackage[headings]{fancyhdr}
\fancyhead[E,O]{}
\fancyfoot[E,O]{}
\renewcommand{\headrulewidth}{0pt}
\pagestyle{fancy}
\setlength{\headsep}{50pt}
\setlength{\textheight}{\textheight-25pt}


\begin{document}

\maketitle
\thispagestyle{empty}

\begin{tabular}[h]{lp{8cm}}
  \textbf{Name, Vorname} & \\\cline{2-2}
  &\\
  \textbf{Matrikelnummer} & \\\cline{2-2}
\end{tabular}

\vspace{2cm}

\begin{center}
  \fbox{\Large{Bearbeiten Sie in Präsenz nur: \textbf{1.1}, \textbf{1.8} und \textbf{alle Teilaufgaben von 2}.}}

  \Zeile

  \Large{Dieser Test dient nur Ihrer eigenen Information.\\

    \Zeile

  Er zeigt Ihnen, welche Voraussetzungen für Ihr Studium\\
  im Bereich \textit{Grammatik} Sie bereits mitbringen und welche nicht.\\

  \Zeile

  Das hier getestete Wissen und die getesteten Fähigkeiten werden\\
  in der Schule oft auf unzulässig simple Weise dargestellt.\\
  Deswegen stellen sie \textbf{keine direkten Studieninhalte} dar,\\
  aber \textbf{wir rechnen trotzdem damit, dass Sie sie mitbringen}.\\
  
  \Zeile

  Einige Fragen zielen auch bereits darauf ab, Ihnen\\
  Unzulänglichkeiten des Schulstoffs deutlich zu machen.
}
\end{center}

\newpage
\setcounter{page}{1}
\fancyfoot[C]{\thepage}
%\fancyhead[RO]{\fbox{\rule[-7pt]{0pt}{20pt}\textbf{Matrikelnummer:} \rule{150pt}{0pt}}}


% https://grammis.ids-mannheim.de/vggf

%%%%%%%%%%%%%%%%%%%%%%%%%%%%%%%%%%%%%%%%%%%%%%%%%%%%%%%%%%%%%%%%%%%%%%%
\section{Wortarten im Deutschen}

\subsection{\fbox{\textbf{Präsenzaufgabe!}} Klassifikation}

Die wichtigsten Wortarten des Deutschen sind die folgenden.
In runden Klammern steht jeweils eine übliche Abkürzung, in eckigen Klammern teilweise gebräuchliche \textbf{deutsche Namen der Wortklassen, die wir im Studium allerdings prinzipiell nicht verwenden}.
Im Studium werden diese Wortarten neu definiert, aber hier geht es erst einmal darum, zu sehen, ob Sie noch wissen, was in der Schule gelehrt wurde.

\begin{itemize}\Lf
  \item \textbf{Substantiv} (Subst) [Hauptwort, Dingwort, Gegenstandswort; auch oft (falsch): Nomen]
  \item \textbf{Adjektiv} (Adj) [Eigenschaftswort, Beiwort, Wie-Wort]
  \item \textbf{Artikel} (Art) [Geschlechtswort, Begleiter]
  \item \textbf{Pronomen} (Pro) [Fürwort]
  \item \textbf{Verb} (V) [Zeitwort, Tun-Wort]
  \item \textbf{Präposition} (Präp) [Beziehungswort, Verhältniswort]
  \item \textbf{Adverb} (Adv) [Umstandswort]
  \item \textbf{neben- und unterordnende Konjunktion} (NK, UK) [Bindewort]
  \item \textbf{Partikel} (Part)
\end{itemize}

Bestimmen Sie die Wortklassen im folgenden Kurztext, indem Sie die entsprechenden Abkürzungen unter die Wörter schreiben.
Für die ersten beiden Wörter wurde das beispielhaft schon erledigt.

\Zeile

\begin{center}
  \large
  \begin{tabular}[h]{|c|c|c|c|c|c|c|c|c|c|}
    \hline
      \textit{Ein} & \textit{Stuhl} & \textit{ist} & \textit{ein} & \textit{nützliches} & \textit{Möbelstück} & \textit{und} & \textit{dient} & \textit{dem} & \textit{Sitzen}. \\
      \hline
      Art & Subst &&&&&&&& \\
    \hline
  \end{tabular}
\end{center}

\Zeile

\begin{center}
  \large
  \begin{tabular}[h]{|c|c|c|c|c|c|c|c|c|c|}
    \hline
      \textit{Oft} & \textit{steht} & \textit{vor} & \textit{ihm} & \textit{ein} & \textit{Tisch}, & \textit{dessen} & \textit{Beine} & \textit{länger} & \textit{sind}. \\
      \hline
      &&&&&&&&& \\
    \hline
  \end{tabular}
\end{center}

\Zeile

\begin{center}
  \large
  \begin{tabular}[h]{|c|c|c|c|c|c|c|c|}
    \hline
      \textit{In} & \textit{Japan} & \textit{sehen} & \textit{traditionelle} & \textit{Tische} & \textit{ganz} & \textit{anders} & \textit{aus}, \\\hline
      &&&&&&& \\\hline
      \textit{weil} & \textit{es} & \textit{dort} & \textit{ja} & \textit{auch} & \textit{keine} & \textit{Stühle} & \textit{gibt}. \\\hline
      &&&&&&& \\
    \hline
  \end{tabular}
\end{center}

\Zeile

\subsection{Substantiv}

Kreuzen Sie die korrekten Aussagen an.

\begin{itemize}[label=\Square]\Lf
  \item Im Plural sind alle Substantive grammatisch weiblich (z.\,B.\ \textit{der Tisch} \ding{222} \textit{die Tische}).
  \item Alle Substantive sind entweder maskulin (grammatisch männlich), feminin (grammatisch weiblich) oder neutral (grammatisch sächlich).
  \item An allen Substantiven wird der Kasus (Fall) immer durch eine spezielle Endung angezeigt.
  \item Fast alle Substantive haben für den Plural eine spezielle Form.
  \item Man kann mehrere Substantive zu einem neuen Substantiv zusammensetzen.
  \item Substantive sind nicht steigerbar.
  \item Substantive bezeichnen Dinge, die man anfassen kann.
  \item Maskuline Substantive können nur männliche Wesen bezeichnen.
  \item Man kann mit zusätzlichen Endungen aus Verben und Adjektiven neue Substantive bilden.
\end{itemize}

\subsection{Adjektiv}

Kreuzen Sie die korrekten Aussagen an.

\begin{itemize}[label=\Square]\Lf
  \item Nach Adjektiven kann man immer mit \textit{Wie ist \ldots\ ?} fragen\\
    (z.\,B.: \textit{der rote Tisch} \ding{222} \textit{Wie ist der Tisch?} \ding{222} \textit{Rot.}).
  \item Adjektive haben ausnahmslos immer ein grammatisches Geschlecht (Genus).
    Das Geschlecht richtet sich nach einem Substantiv.
  \item Alle Adjektive bezeichnen Eigenschaften von Substantiven.
  \item Adjektive haben besondere Formen, je nachdem, ob ein bestimmter oder unbestimmter Artikel vor ihnen steht.
  \item Adjektive sind inhaltlich ausschmückend und können daher immer weggelassen werden, ohne dass sich die Aussage des Satzes ändert.
  \item Adjektive können auch wie Substantive verwendet werden, wenn kein Substantiv nach ihnen steht.
  \item Prädikative Adjektive treten immer zusammen mit der Form eines Verbs wie \textit{sein}, \textit{bleiben}, \textit{werden} auf.
\end{itemize}

\subsection{Artikel}

Kreuzen Sie die korrekten Aussagen an.

\begin{itemize}[label=\Square]\Lf
  \item Artikel sind dazu da, das grammatische Geschlecht (Genus) des Substantivs anzuzeigen.
    Deswegen heißen sie in der Grundschuldidaktik (leicht veraltet) Geschlechtswort.
  \item Artikel stehen immer vor einem Substantiv und stimmen mit diesem im Numerus (Singular\slash Plural) und dem Fall (Kasus) überein.
  \item Alle Artikel haben jeweils spezifische Formen für die beiden Numeri (z.\,B.\ \textit{der Tisch} \ding{222} \textit{die Tische}).
\end{itemize}

% definit vs. indefinit

\subsection{Pronomen}

Kreuzen Sie die korrekten Aussagen an.

\begin{itemize}[label=\Square]\Lf
  \item Pronomina ersetzen immer ein Substantiv.
  \item Pronomina haben spezifische Formen für die Numeri (Singular\slash Plural).
  \item Pronomina und Artikel sind die einzigen Wortklassen im Deutschen, an deren Mitgliedern man alle vier Kasus (Fälle) unterscheiden kann.
  \item Das Wort \textit{viel} wie in \textit{die vielen Erdbeeren} ist ein Indefinitpronomen.
  \item Das Personalpronomen hat spezifische Formen für die drei grammatischen Personen im Singular.
  \item Das Demonstrativpronomen hat spezifische Formen für die drei grammatischen Personen im Singular.
\end{itemize}

\subsection{Verb}

Kreuzen Sie die korrekten Aussagen an.

\begin{itemize}[label=\Square]\Lf
  \item Starke Verben verändern in der einfachen Vergangenheitsform (Präteritum) ihren Vokal.
  \item Verben beschreiben immer Handlungen (\textit{essen}, \textit{kaufen}, \textit{vereinbaren} usw.).
  \item Verben müssen immer in ein Tempus (Zeitform) gesetzt werden (\textit{ich gehe}, \textit{ich ging} usw.).
  \item Transitive Verben treten mit einem Subjekt und einem Akkusativobjekt auf.
  \item Nur transitive Verben kann man ins Passiv setzen\\
    (z.\,B.\ \textit{Wir kaufen den Saft. \ding{222}\ Der Saft wird gekauft.).}
  \item Intransitive Verben haben kein Akkusativobjekt.
  \item Das Verb \textit{sein} (\textit{ich bin} usw.) ist unregelmäßig.
  \item Modalverben (\textit{müssen}, \textit{können} usw.) treten immer zusammen mit einem anderen Verb auf.
  \item Nach Verben kann man fragen mit \textit{Was macht\slash tut \ldots\ ?}
    Deswegen heißen sie in der Grundschuldidaktik Tun-Wörter.
  \item Hilfsverben werden unter anderem benutzt, um Tempora (Zeitformen) auszudrücken.
  \item Es gibt einen Infinitiv des Perfekts.
\end{itemize}

\Zeile

\subsection{Präposition}

Kreuzen Sie die korrekten Aussagen an.

\begin{itemize}[label=\Square]\Lf
  \item Präpositionen bestimmen ein Substantiv näher (z.\,B.\ \textit{unter dem Tisch}).
  \item Präpositionen fordern immer einen bestimmten Kasus (Fall) beim Substantiv, das ihnen folgt.
  \item Präpositionen bilden immer adverbiale Bestimmungen und können weggelassen werden.
  \item Manche Präpositionen können je nach Bedeutung entweder den Dativ oder den Akkusativ fordern.
\end{itemize}

\Zeile

\subsection{\fbox{\textbf{Präsenzaufgabe!}} Unterklassifikation von Verben}

Klassifizieren Sie die unterstrichenen Verben als starke Verben, schwache Verben, Modalverben oder Hilfsverben.

\begin{center}
  \begin{tabular}[h]{cp{0.5\textwidth}cccc}
    \toprule
    & \textbf{Verb im Satzkontext} & \multicolumn{4}{l}{\textbf{Bestimmung}} \\
    \midrule
    (1) & \textit{Marjella und ihre Freunde \ul{laufen} sehr schnell.}  & \Square~stark & \Square~schwach & \Square~Modalv. & \Square~Hilfsv. \\
    (2) & \textit{In den Urlaub \ul{wollten} 2020 viele fahren.}        & \Square~stark & \Square~schwach & \Square~Modalv. & \Square~Hilfsv. \\
    (3) & \textit{Wir \ul{kaufen} viel zu viel unnützes Zeug.}          & \Square~stark & \Square~schwach & \Square~Modalv. & \Square~Hilfsv. \\
    (4) & \textit{Du \ul{wirst} bald in den Urlaub fahren.}             & \Square~stark & \Square~schwach & \Square~Modalv. & \Square~Hilfsv. \\
    (5) & \textit{Es ist gut, dass sie wieder laufen \ul{kann}.}        & \Square~stark & \Square~schwach & \Square~Modalv. & \Square~Hilfsv. \\
    (6) & \textit{\ul{Durchschwimmen} kann man den Ärmelkanal auch.}    & \Square~stark & \Square~schwach & \Square~Modalv. & \Square~Hilfsv. \\
  \end{tabular}
\end{center}

\newpage

%%%%%%%%%%%%%%%%%%%%%%%%%%%%%%%%%%%%%%%%%%%%%%%%%%%%%%%%%%%%%%%%%%%%%%%
\section{Flexionskategorien deutscher Wörter}

\subsection{\fbox{\textbf{Präsenzaufgabe!}} Flexion (Beugung)}

Bilden Sie die genannten Formen der unten in ihrer jeweiligen Nennform angegebenen Wörter.
Hinweis: Mit Präteritum bezeichnet man die einfache Vergangenheitsform.

\begin{center}
  \begin{tabular}[h]{cllp{0.3\textwidth}}
    \toprule
    & \textbf{Wort} & \textbf{zu bildende Form} & \textbf{Form} \\
    \midrule
    &&& \\
    (1) & \textit{fechten} & 3.~Person Singular Indikativ Präsens Aktiv    & \\\cline{4-4}
    &&& \\
    (2) & \textit{Haus}    & Dativ Plural                                  & \\\cline{4-4}
    &&& \\
    (3) & \textit{laufen}  & 2.~Person Singular Indikativ Präteritum Aktiv & \\\cline{4-4}
    &&& \\
    (4) & \textit{dies}    & Femininum Genitiv Singular                    & \\\cline{4-4}
    &&& \\
    (5) & \textit{Oma}     & Genitiv Singular                              & \\\cline{4-4}
    &&& \\
    (6) & \textit{streichen} & 3.~Person Plural Indikativ Futur 1 Passiv   & \\\cline{4-4}
    \end{tabular}
\end{center}

\subsection{\fbox{\textbf{Präsenzaufgabe!}} Kasus (Fall)}

Bestimmen Sie die Kasus -- also Nominativ, Akkusativ, Dativ, Genitiv -- der unterstrichenen Wörter.

\begin{center}
  \begin{tabular}[h]{cp{0.55\textwidth}cccc}
    \toprule
    & \textbf{Wort im Satzkontext} & \multicolumn{4}{l}{\textbf{Kasus}} \\
    \midrule
    (1) & \textit{\ul{Menschen} glauben wir oft zu leichtfertig.}      & \Square~Nom & \Square~Akk & \Square~Dat & \Square~Gen \\
    (2) & \textit{Günther lobt \ul{meinen} Fahrstil.}                   & \Square~Nom & \Square~Akk & \Square~Dat & \Square~Gen \\
    (3) & \textit{Selten wird das \ul{Auto} mehr als 200 km gefahren.} & \Square~Nom & \Square~Akk & \Square~Dat & \Square~Gen \\
    (4) & \textit{\ul{Es} wird deutlich zu viel Energie verbraucht.}   & \Square~Nom & \Square~Akk & \Square~Dat & \Square~Gen \\
    (5) & \textit{Das ist die Vorschrift, \ul{der} wir gehorchen.}     & \Square~Nom & \Square~Akk & \Square~Dat & \Square~Gen \\
    (6) & \textit{Das Auto der \ul{Kollegin} streikt mal wieder.}      & \Square~Nom & \Square~Akk & \Square~Dat & \Square~Gen \\
  \end{tabular}
\end{center}

\subsection{\fbox{\textbf{Präsenzaufgabe!}} Genus (grammatisches Geschlecht)}

Bestimmen Sie das Genus -- also Maskulinum, Neutrum oder Femininum -- der unterstrichenen Wörter.

\begin{center}
  \begin{tabular}[h]{cp{0.55\textwidth}ccc}
    \toprule
    & \textbf{Wort im Satzkontext} & \multicolumn{3}{l}{\textbf{Kasus}} \\
    \midrule
    (1) & \textit{\ul{Der} Quark hält sich noch länger.}         & \Square~Mask & \Square~Neut & \Square~Fem \\
    (2) & \textit{\ul{Der} Kollegin gefällt das neue Büro.}      & \Square~Mask & \Square~Neut & \Square~Fem \\
    (3) & \textit{\ul{Der} Lämmer Fell ist weich.}               & \Square~Mask & \Square~Neut & \Square~Fem \\
    (4) & \textit{Dan sammelt kunstvolle \ul{Keramikkrüge}.}     & \Square~Mask & \Square~Neut & \Square~Fem \\
    (5) & \textit{Und reinigt die \ul{Tröge} gut!}               & \Square~Mask & \Square~Neut & \Square~Fem \\
    (6) & \textit{Wie diese \ul{Sykophanten} mal wieder nerven!} & \Square~Mask & \Square~Neut & \Square~Fem \\
  \end{tabular}
\end{center}


\subsection{\fbox{\textbf{Präsenzaufgabe!}} Person}

Bestimmen Sie die Person -- also 1, 2 oder 3 -- der unterstrichenen Wörter (bzw. Wortgruppen im Fall von \textit{Herrn Gödel}).


\begin{center}
  \begin{tabular}[h]{cp{0.55\textwidth}ccc}
    \toprule
    & \textbf{Wort im Satzkontext} & \multicolumn{3}{l}{\textbf{Person}} \\
    \midrule
    (1) & \textit{Ich wünschte, du \ul{höbst} den Schwamm auf.}               & \Square~1 & \Square~2 & \Square~3 \\
    (2) & \textit{Mir gefällt \ul{euer} Haus sehr.}                           & \Square~1 & \Square~2 & \Square~3 \\
    (3) & \textit{Ich sehe \ul{es}.}                                          & \Square~1 & \Square~2 & \Square~3 \\
    (4) & \textit{Meine ehemalige Kollegin heißt \ul{Marjella}}               & \Square~1 & \Square~2 & \Square~3 \\
    (5) & \textit{\ul{Ich} bin Elektrotechniker.}                             & \Square~1 & \Square~2 & \Square~3 \\
    (6) & \textit{Sein Genie hat \ul{Herrn Gödel} den Verstand gekostet.}     & \Square~1 & \Square~2 & \Square~3 \\
    (7) & \textit{Dass Strom billiger würde, \ul{möchte} mir schon gefallen.} & \Square~1 & \Square~2 & \Square~3 \\
  \end{tabular}
\end{center}

\subsection{\fbox{\textbf{Präsenzaufgabe!}} Finitheit}

\ul{Unterstreichen} Sie im folgenden Text alle finiten Verbformen und \mybox{rahmen} Sie alle infiniten Verbformen ein. (Quelle: https://de.wikipedia.org/wiki/Sokal-Affäre, modifiziert)

\begin{quote}
  \onehalfspacing
  \it Die Sokal-Affäre (auch Sokal-Debatte oder Sokal-Kontroverse) war eine Auseinandersetzung über die intellektuellen Standards in den Sozial- und Geisteswissenschaften, die durch die Veröffentlichung eines Hoax-Artikels des Physikers Alan Sokal in der sozialwissenschaftlichen Fachzeitschrift Social Text ausgelöst wurde. Sokals Artikel erschien 1996 in einer den Science Wars (Wissenschaftskriegen) gewidmeten Ausgabe, die die US-spezifische Auseinandersetzung zwischen wissenschaftlichem Realismus und Postmoderne thematisieren sollte.\\
Sokals Beitrag war in postmodernem Jargon formuliert und gab vor, die Quantengravitation als linguistisches und soziales Konstrukt zu deuten, wobei die Quantenphysik die postmodernistische Kritik stütze. Sokal hatte dabei absichtlich zahlreiche logische und inhaltliche Fehler eingestreut, die den Redakteuren der Zeitschrift – sie hatten für die Schlussredaktion keine Physikexperten hinzugezogen – jedoch nicht auffielen. Es folgte eine wissenschaftstheoretische und öffentliche Debatte über mangelnde intellektuelle Strenge bei der Bewertung pseudowissenschaftlicher Artikel in den Sozial- und Geisteswissenschaften und einen möglicherweise schädlichen Einfluss postmoderner Philosophie auf diese Wissenschaften. Weiterhin wurde diesen Disziplinen vorgeworfen, naturwissenschaftliche Konzepte in sinnloser oder missbräuchlicher Weise für ihre Lehren zu verwenden.
\end{quote}

\subsection{\fbox{\textbf{Präsenzaufgabe!}} Tempus (Zeitform)}

Bestimmen Sie das Tempus der folgenden Sätze.
Die Tempuskategorien im weiteren Sinn sind im Deutschen: \textbf{Präsens} (Gegenwart), \textbf{Präteritum} (einfache Vergangenheit), \textbf{Perfekt} (vollendete Vergangenheit), \textbf{Plusquamperfekt} (Vorvergangenheit), \textbf{Futur 1} (einfache Zukunft) und \textbf{Futur 2} (Futurum Exactum oder Vergangenheit in der Zukunft).
Verwenden Sie ggf.\ die Abkürzungen \textbf{Präs}, \textbf{Prät}, \textbf{Perf}, \textbf{Plusq}, \textbf{F1}, \textbf{F2}.

\begin{center}
  \begin{tabular}[h]{clp{0.3\textwidth}}
    \toprule
    & \textbf{Wort im Satzkontext} & \textbf{Tempus} \\
    \midrule
    && \\
    (1) & \textit{Der Schwan wird ungern fotografiert.}                     &  \\ \cline{3-3}
    && \\
    (2) & \textit{Mir war übel geworden.}                                   &  \\ \cline{3-3}
    && \\
    (3) & \textit{Im  Jahr 2050 wird Helmut Schmidt abgewählt worden sein.} &  \\ \cline{3-3}
    && \\
    (4) & \textit{Morgen gehe ich endlich zur Post!}                        &  \\ \cline{3-3}
    && \\
    (5) & \textit{Der Schwan wird ungern fotografiert werden.}              &  \\ \cline{3-3}
    && \\
    (6) & \textit{Das Boot hat auslaufen können.}                           &  \\ \cline{3-3}
    && \\
    (7) & \textit{1993 hat der Kommerz den Techno zerstört.}                &  \\ \cline{3-3}
  \end{tabular}
\end{center}

\Doppelzeile

\subsection{\fbox{\textbf{Präsenzaufgabe!}} Modus (Aussageweise)}

Setzen Sie den folgenden kurzen Absatz zunächst in den Konjunktiv 1, dann in den Konjunktiv 2.
Folgen Sie den Normregeln für die Ersetzung von Formen, die ansonsten nicht eindeutig wären.

\begin{quote}
  Die Grammatik folgt Regeln, und sie folgte schon immer Regeln.
  Nur das kann der Grund sein, dass wir einander verstehen, wenn wir Sprache benutzen.
  Die Mathematik ist axiomatisch eingeführt worden.
  Sie gehorcht damit ausnahmslosen Regeln, während die Regeln der Grammatik Ausnahmen zulassen.
\end{quote}


\begin{center}
  \begin{tabular}[h]{cp{0.8\textwidth}}
  &\\
  \textbf{Im Konjunktiv 1:} &\\\cline{2-2}
  &\\
  &\\\hline
  &\\
  &\\\hline
  &\\
  &\\\hline
  &\\
  &\\\hline
  &\\
  &\\\hline
  &\\
  &\\\hline
  \end{tabular}
\end{center}

\begin{center}
  \begin{tabular}[h]{cp{0.8\textwidth}}
  &\\
  \textbf{Im Konjunktiv 2:} &\\\cline{2-2}
  &\\
  &\\\hline
  &\\
  &\\\hline
  &\\
  &\\\hline
  &\\
  &\\\hline
  &\\
  &\\\hline
  &\\
  &\\\hline
  \end{tabular}
\end{center}


\subsection{\fbox{\textbf{Präsenzaufgabe!}} Imperativ (Aufforderungsform)}

Kreuzen Sie die Sätze an, in denen eine echte Imperativform vorkommt.

\begin{itemize}[label=\Square]\Lf
  \item \textit{Komm du mir nur nach hause!}
  \item \textit{Ich möchte, dass du sofort dein Zimmer aufräumst.}
  \item \textit{Den Eischnee langsam mit einer Gabel unterheben.}
  \item \textit{Wirst du wohl die anderen Kinder in Ruhe lassen!}
  \item \textit{Nimm dir bitte Schokolade, soviel du möchtest.}
  \item \textit{Stehengeblieben!}
  \item \textit{Hier wird nicht geraucht!}
  \item \textit{Ich befehle dir, sofort mit dem Hund rauszugehen.}
  \item \textit{Liegenbleiben!}
  \item \textit{Glaubt bloß nicht, dass die Krankenkasse das bezahlt.}
  \item \textit{Gibst du das wohl sofort deiner Schwester zurück!}
\end{itemize}

\subsection{\fbox{\textbf{Präsenzaufgabe!}} Genus verbi (Aktiv\slash Passiv)}

Setzen Sie die folgenden Sätze ins Aktiv, wenn es Passivsätze sind, und ins Passiv, wenn es Aktivsätze sind.

\begin{center}
  \begin{tabular}[h]{cp{0.9\textwidth}}
    & \\
    (1) & \textit{Ein Kollege gibt mir das Buch.} \\
    & \\
    &\\\cline{2-2}
    & \\
    &\\\cline{2-2}
    & \\
    (2) & \textit{Der Kuchen wurde von unserem Hund gegessen.} \\
    & \\
    &\\\cline{2-2}
    & \\
    &\\\cline{2-2}
    & \\
  \end{tabular}
\end{center}

\newpage

\begin{center}
  \begin{tabular}[h]{cp{0.9\textwidth}}
    & \\
    (3) & \textit{Der Kuchen ist von unserem Hund gegessen worden.} \\
    & \\
    &\\\cline{2-2}
    & \\
    &\\\cline{2-2}
    & \\
    (4) & \textit{Man kauft hier gerne Limo.} \\
    & \\
    &\\\cline{2-2}
    & \\
    &\\\cline{2-2}
    & \\
    (5) & \textit{Hier wird nicht geraucht!} \\
    & \\
    &\\\cline{2-2}
    & \\
    &\\\cline{2-2}
  \end{tabular}
\end{center}


\section{Satzbau}

\subsection{Satzglieder}

Zeichnen Sie einen \mybox{Kasten} um jedes Satzglied in folgenden Sätzen.

\begin{exe}
  \setcounter{xnumi}{0}
  \ex \textit{Menschen glauben wir oft zu leichtfertig.}\\

  \ex \textit{Günther lobt meinen Fahrstil.}\\

  \ex \textit{Selten wird das Auto mehr als 200 km gefahren.}\\

  \ex \textit{Es wird deutlich zu viel Energie verbraucht.}\\

  \ex \textit{Das ist die Vorschrift, der wir gehorchen.}\\

  \ex \textit{Das Auto der Kollegin streikt mal wieder.}
\end{exe}

\subsection{Subjekt}

Unterstreichen Sie das Subjekt in den folgenden Sätzen.

\begin{exe}
  \setcounter{xnumi}{0}
  \ex Dass die Welt vergänglich ist, weiß ich.\\

  \ex Gestern hatte der Kollege das Buch noch gesehen.\\

  \ex Dass die Welt vergänglich ist, ist mir bekannt.\\

  \ex Es gehen mir hier zu viele Leute über die Straße.\\

  \ex Den Mülleimer zu leeren, nervt Matthias.\\

  \ex Uns graut vor den neuen Quartalszahlen.\\

  \ex Das Auto fährt mir die Oma zu oft zu schnell.\\

\end{exe}

\subsection{Objekte und adverbiale Bestimmungen}

\textbf{\ul{Unterstreichen}} Sie im folgenden Text die direkten Objekte in den folgenden Sätzen und \textbf{\ol{überstreichen}} Sie in denselben Sätzen alle indirekten Objekte. Die Präpositionalobjekte \textbf{\mybox{rahmen}} Sie ein.
Die adverbialen Bestimmungen \textbf{(klammern)} Sie bitte ein.
(Quelle: https://de.wikipedia.org/wiki/Schlacht\_von\_Worringen, modifiziert)

\begin{quote}
  \onehalfspacing
  Die Schlacht von Worringen war 1288 das kriegerische Finale im zuvor bereits sechs Jahre währenden Limburger Erbfolgestreit.
  Hauptkontrahenten des Konflikts waren Siegfried von Westerburg, Erzbischof von Köln, und Herzog Johann I. von Brabant.
  Der Ausgang der Schlacht veränderte das Machtgefüge im gesamten Nordwesten Mitteleuropas.\\
  Der Ausgang der Schlacht hatte für jede der involvierten Parteien erhebliche Konsequenzen.
  Erzbischof Siegfried von Westerburg befand sich als Gefangener in der Gewalt des Grafen von Berg im "`Novum Castrum"'.
  Erst durch den Sühnevertrag vom 19. Mai 1289 erlangte er die Freiheit wieder.
  Inzwischen hatte der Dompropst von Köln, Konrad von Berg, ein Bruder von Adolf von Berg, die Regierungsgewalt des Erzstifts übernommen.
  Die Gewinner der Schlacht hatten Tatsachen geschaffen, die Siegfried neben der Lösegeldzahlung von 12.000 Mark wohl oder übel durch den Sühnevertrag billigen musste.
  Außerdem musste er auf sein Befestigungsrecht im Bergischen Land verzichten.
  Eberhard von der Mark erhielt Befestigungshoheit und Adolf von Berg sein Münzrecht, auf das er 1279 zugunsten des Erzbischofs hatte verzichten müssen, zurück.
\end{quote}

%\subsection{Satzgliedstellung}

\subsection{Nebensätze}

Bestimmen Sie die Nebensätze in den folgenden Sätzen als Subjektsatz, Objektsatz, Adverbialsatz oder Relativsatz.

\begin{center}
  \begin{tabular}[h]{cp{0.5\textwidth}cccc}
    \toprule
    & \textbf{Satz mit Nebensatz} & \multicolumn{4}{l}{\textbf{Nebensatzart}} \\
    \midrule
    (1) & \textit{Damit es nicht zu spät wird, gehen wir jetzt.}    & \Square~Subj & \Square~Obj & \Square~Adv & \Square~Rel \\
    (1) & \textit{Wer das glaubt, hat keine Ahnung von Physik.}     & \Square~Subj & \Square~Obj & \Square~Adv & \Square~Rel \\
    (1) & \textit{Ob die Sonne scheinen wird, ist die große Frage.} & \Square~Subj & \Square~Obj & \Square~Adv & \Square~Rel \\
    (1) & \textit{Marjella freut, dass die Sonne scheint.}          & \Square~Subj & \Square~Obj & \Square~Adv & \Square~Rel \\
    (1) & \textit{Wir fragen uns, ob das Wetter heute gut wird.}    & \Square~Subj & \Square~Obj & \Square~Adv & \Square~Rel \\
    (1) & \textit{Das ist der Kollege, dessentwegen ich hier bin.}  & \Square~Subj & \Square~Obj & \Square~Adv & \Square~Rel \\
  \end{tabular}
\end{center}

\subsection{Infinitivgruppen}

Unterstreichen Sie die kompletten Infinitivgruppen (auch: erweiterte Infinitive) in folgenden Sätzen.
Bestimmen Sie außerdem ihre Funktion im Satz: Subjekt, Objekt oder Adverbial.

\Zeile

\begin{center}
  \begin{tabular}[h]{cp{0.65\textwidth}ccc}
    \toprule
    & \textbf{Infinitivgr.\ im Satzkontext} & \multicolumn{3}{l}{\textbf{Bestimmung}} \\
    \midrule
    (1) & \textit{Den Stuhl zu reparieren, mag Matthias nicht.}                & \Square~Subj & \Square~Obj & \Square~Adv \\
    &&&& \\
    (2) & \textit{Den Stuhl zu reparieren, nervt.}                             & \Square~Subj & \Square~Obj & \Square~Adv \\
    &&&& \\
    (3) & \textit{Um den Stuhl zu reparieren, geht Matthias in die Werkstatt.} & \Square~Subj & \Square~Obj & \Square~Adv \\
    &&&& \\
    (4) & \textit{Sie wagt, die Küche zu betreten.}                            & \Square~Subj & \Square~Obj & \Square~Adv \\
    &&&& \\
    (5) & \textit{Er stellt die Karre ab, ohne den Lackschaden zu erwähnen.}   & \Square~Subj & \Square~Obj & \Square~Adv \\
    &&&& \\
    (6) & \textit{Es wurde versucht, die Demonstration zu verhindern.}         & \Square~Subj & \Square~Obj & \Square~Adv \\
    &&&& \\
  \end{tabular}
\end{center}

\Zeile

% %%%%%%%%%%%%%%%%%%%%%%%%%%%%%%%%%%%%%%%%%%%%%%%%%%%%%%%%%%%%%%%%%%%%%%%
\section{Buchstaben und Laute}
 
\subsection{Laute und Buchstaben}

Welche der unterstrichenen Buchstaben oder Buchstabengruppen in den folgenden Wortpaaren werden in beiden Wörtern gleich ausgesprochen?

\Zeile

\begin{center}
  \begin{tabular}[h]{rllcc}
    \toprule
    & \textbf{Wort 1} & \textbf{Wort 2} & \multicolumn{2}{l}{\textbf{Aussprache}} \\
    \midrule
    (1) & ba\ul{t}      & Ba\ul{d}         & \Square~gleich & \Square~nicht gleich \\
    (2) & wei\ul{ch}en  & wa\ul{ch}en      & \Square~gleich & \Square~nicht gleich \\
    (3) & R\ul{o}be     & R\ul{o}bbe       & \Square~gleich & \Square~nicht gleich \\
    (4) & \ul{k}lein    & ha\ul{ck}en      & \Square~gleich & \Square~nicht gleich \\ 
    (5) & \ul{L}and     & Ba\ul{ll}        & \Square~gleich & \Square~nicht gleich \\
    (6) & sp\ul{ä}ter   & \ul{Eh}re        & \Square~gleich & \Square~nicht gleich \\
    (7) & kl\ul{ar}     & F\ul{ah}ne       & \Square~gleich & \Square~nicht gleich \\
    (8) & \ul{r}ar      & ra\ul{r}         & \Square~gleich & \Square~nicht gleich \\
    (9) & R\ul{eh}      & Schn\ul{ee}      & \Square~gleich & \Square~nicht gleich \\
    (10) & frü\ul{h}er  & \ul{h}art        & \Square~gleich & \Square~nicht gleich \\
  \end{tabular}
\end{center}

\newpage

\subsection{Silben}\label{sec:silben}

Trennen Sie die folgenden Wörter in Silben.
Nutzen Sie dazu wie in Beispiel (0) demonstriert Punkte als Trenner.
Für das erste Wort gibt es eine Lösung als Beispiel.

\begin{center}
  \begin{tabular}[h]{clp{0.5\textwidth}}
    && \\
    (0) & \textit{Tinte} & Tin.te \\ \cline{3-3}
    && \\
    (1) & \textit{verwundert} & \\ \cline{3-3}
    && \\
    (2) & \textit{Desorientierung} & \\ \cline{3-3}
    && \\
    (3) & \textit{Wege} & \\ \cline{3-3}
    && \\
    (4) & \textit{Automat} & \\ \cline{3-3}
    && \\
    (5) & \textit{Anklang} & \\ \cline{3-3}
    && \\
    (6) & \textit{Politik} & \\ \cline{3-3}
    && \\
    (7) & \textit{Iglo} & \\ \cline{3-3}
    && \\
    (8) & \textit{Anschrift} & \\ \cline{3-3}
    && \\
    (9) & \textit{Küchen} & \\ \cline{3-3}
    && \\
    (10) & \textit{munter} & \\ \cline{3-3}
    && \\
    (11) & \textit{strolchtest} & \\ \cline{3-3}
    && \\
    (12) & \textit{klapprigstes} & \\ \cline{3-3}
    && \\
    (13) & \textit{Marmelade} & \\ \cline{3-3}
    && \\
    (14) & \textit{Mangel} & \\ \cline{3-3}
    && \\
    (15) & \textit{Metropolis} & \\ \cline{3-3}
  \end{tabular}
\end{center}

\Zeile

\subsection{Betonung}

Setzen Sie in Aufgabe~\ref{sec:silben} einen Akutakzent (also das Zeichen \'\ \,) über den Vokal der betonten Silbe in den von Ihnen in Silben zerlegten Wörtern in Aufgabe~\ref{sec:silben}.
Also \textbf{T\'in.te} usw.

\end{document}
